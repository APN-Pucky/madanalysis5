\documentclass[a4paper]{article}

%% Language and font encodings
\usepackage[english]{babel}
\usepackage[utf8x]{inputenc}
\usepackage[T1]{fontenc}
\usepackage{url,slashed,multirow,cite,verbdef,hyperref,amsmath,graphicx,xcolor}
\hypersetup{bookmarks=true,unicode=true,pdftoolbar=false,pdfmenubar=true,
  pdffitwindow=false,pdfstartview={FitH},pdftitle={MA5 Normal Reference Card},
  pdfauthor={J.~Araz, E.~Conte, B.~Fuks}, pdfsubject={MA5 Normal Reference Card},
  pdfcreator={J.~Araz, E.~Conte, B.~Fuks},pdfproducer={PDFLaTeX}, pdfkeywords={MA5normal-mode},
  pdfnewwindow=true,colorlinks=true,
  linkcolor=blue,citecolor=magenta,filecolor=magenta,urlcolor=cyan}

%% Sets page size and margins
\usepackage[a4paper,top=3cm,bottom=2cm,left=3cm,right=3cm,marginparwidth=1.75cm]{geometry}


% boxes
\usepackage{framed,color}
\definecolor{ao}{rgb}{0.0, 0.0, 0.55}
\definecolor{shadecolor}{rgb}{0.4, 0.6, 0.8}

% Tools
\newcommand{\MA}{\textsc{MadAnalysis}~5}
\newcommand{\MAnorm}{{MadAnalysis}~5}
\newcommand{\FJ}{\textsc{FastJet}}
\newcommand{\FJnorm}{{FastJet}}
\newcommand{\DEL}{\textsc{Delphes}}
\newcommand{\DELnorm}{{Delphes}}
\newcommand{\ROOT}{\textsc{Root}}
\newcommand{\lhe}{\textsc{Lhe}}
\newcommand{\lhco}{\textsc{Lhco}}
\newcommand{\hepmc}{\textsc{HepMC}}
\newcommand{\stdhep}{\textsc{StdHep}}
\newcommand{\INSP}{\textsc{Inspire}}
\newcommand{\python}{\textsc{Python}}
\newcommand{\spla}{\textsc{SampleAnalyzer}}
%latin
\newcommand{\eg}{\textit{e.g.}}
\newcommand{\ie}{\textit{i.e.}}
\newcommand{\etc}{\textit{etc.}}

% environments
\def\be{\begin{equation}}
\def\ee{\end{equation}}
\def\bsp#1\esp{\begin{split}#1\end{split}}

\begin{document}
\begin{center} 
{\color{ao} \huge \textbf{\textsc{MadAnalysis} 5 v1.8}} \\ \vspace{0.25cm}
{\color{ao} \huge \textbf{Normal-Mode reference card}} \\ \vspace{0.25cm}
\vspace{0.1cm} \textsl{\today}\\
\vspace{0.1cm} \textsl{website: \url{http://madanalysis.irmp.ucl.ac.be/}}\\
\vspace{0.1cm} \textsl{References:\\ \href{https://arxiv.org/abs/1206.1599}{arXiv:1206.1599 [hep-ph]}, \href{https://arxiv.org/abs/1405.3982}{arXiv:1405.3982 [hep-ph]}, 
\href{https://arxiv.org/abs/1407.3278}{arXiv:1407.3278 [hep-ph]},\\
\href{https://arxiv.org/abs/1910.11418}{arXiv:1910.11418 [hep-ph]},
\href{https://arxiv.org/abs/2006.09387}{arXiv2006.09387 [hep-ph]}
}
\end{center}

\hspace{0.5cm}

%%%%%%%%%%%%%%%%%%%%%%%%%%%%%%%%%%%%%%%%%%%%%%%%%%%%%%%%%%%%%%%%%%%%%%%%%%%%%%%%%%%%%%%%%%%%%%
\begin{shaded}
\section{\Large Starting the \MAnorm\ interpreter}
\end{shaded}
%%%%%%%%%%%%%%%%%%%%%%%%%%%%%%%%%%%%%%%%%%%%%%%%%%%%%%%%%%%%%%%%%%%%%%%%%%%%%%%%%%%%%%%%%%%%%%

\noindent The \MA\ command line interface can be started by typing in a shell\\
{\color{ao}\begin{verbatim} bin/ma5 [options] [script] \end{verbatim}}
\noindent where the potential options ({\color{ao}\verb+[options]+}) are given in the table below.
\verbdef{\vv}{-v}
\renewcommand{\arraystretch}{1.2}%
\begin{center}\begin{tabular}{ l l  p{8.1cm}}
\hline
Short & Full & Description\\
\hline
\color{ao}\verb?-P? & \color{ao}\verb?--partonlevel? & Parton-level mode.\\
\color{ao}\verb?-H? & \color{ao}\verb?--hadronlevel? & Hadron-level mode.\\
\color{ao}\verb?-R? & \color{ao}\verb?--recolevel?   & Reconstructed-level mode.\\
\color{ao}\verb?-E? & \color{ao}\verb?--expert?      & Expert mode.\\
\multirow{2}{*}{\color{ao}\vv} & \color{ao}\verb?--version?
          & \multirow{2}{*}{Displays the current \MA\ version number.}\\
          &  \color{ao}\verb?--release?    & \\
\color{ao}\verb?-b? & \color{ao}\verb?--build?       & Builds of the \spla\ library.\\
\color{ao}\verb?-f? & \color{ao}\verb?--forced?      & Skips \MA\ confirmation messages.\\
\color{ao}\verb?-s? & \color{ao}\verb?--script?      & Executes a script containing all analysis
    commands and exits the program. The file containing the script has to be
    provided as {\color{ao}\verb+[script]+}. Removing the {\color{ao}\verb+-s+}
    option prevents \MA\ from exiting.\\
\color{ao}\verb?-h? & \color{ao}\verb?--help?        & Print all the available commands.\\
\color{ao}\verb?-i? & \color{ao}\verb?--installcard? & Generates a file with information on the
     \MA\ dependencies, \verb?installation_card.dat?, that is located in the
     \verb+madanalysis/input+ folder.\\
\color{ao}\verb?-d? & \color{ao}\verb?--debug?       & Debug mode.\\
\hline
\end{tabular}
\end{center}
In case of installation issues, the usage of the dependencies can be tuned by
editing the file:\\
{\centering \color{ao} \verb?madanalysis/input/installation_options.dat?}\\
\noindent More information can be found in the original \MA\ manual~\cite{Conte:2012fm}.
\newpage

%%%%%%%%%%%%%%%%%%%%%%%%%%%%%%%%%%%%%%%%%%%%%%%%%%%%%%%%%%%%%%%%%%%%%%%%%%%%%%%%%%%%%%%%%%%%%%
\begin{shaded}
\section{\Large Installation of optional packages}
\label{sec:install}\end{shaded}
%%%%%%%%%%%%%%%%%%%%%%%%%%%%%%%%%%%%%%%%%%%%%%%%%%%%%%%%%%%%%%%%%%%%%%%%%%%%%%%%%%%%%%%%%%%%%%

\noindent \MA\ is interfaced to several high-energy physics packages and can be linked to
a variety of external plugins. Any of those can be installed by typing in the
\MA\ interpreter,
{\color{ao}\begin{verbatim} install [package]\end{verbatim}}
where the different choices for \verb+[package]+ are given in the table below.
\renewcommand{\arraystretch}{1.2}%
\begin{center}\begin{tabular}{l p{9.2cm}}
\hline
Package & Description\\
\hline
\color{ao} \verb?PAD?            & The \MA\ Public Analysis Database of recasted LHC
   analyses relying on \DEL~3 for the simulation of the detector effects.\\
\color{ao}\verb?PADforMA5tune?  & The \MA\ Public Analysis Database of recasted LHC
   analyses relying on the old tuned version of \DEL~3 for the simulation of the
   detector effects.\\
\color{ao}\verb?delphes?        & The current release of \DEL~3.\\
\color{ao}\verb?delphesMA5tune? & The old tuned version of \DEL~3.\\
\color{ao}\verb?fastjet?        & The \FJ\ and {\sc FastJetContrib} packages.\\
\color{ao}\verb?samples?        & Some test Monte Carlo samples.\\
\color{ao}\verb?zlib?           & The {\sc ZLib} library allowing to handle gzipped
                        compressed files.\\
\hline
\end{tabular}
\end{center}
\vspace{2cm}
%%%%%%%%%%%%%%%%%%%%%%%%%%%%%%%%%%%%%%%%%%%%%%%%%%%%%%%%%%%%%%%%%%%%%%%%%%%%%%%%%%%%%%%%%%%%%%
\begin{shaded}
\section{\Large Commands available from the \MAnorm\ interpreter}
\end{shaded}
%%%%%%%%%%%%%%%%%%%%%%%%%%%%%%%%%%%%%%%%%%%%%%%%%%%%%%%%%%%%%%%%%%%%%%%%%%%%%%%%%%%%%%%%%%%%%%

\noindent A small set of commands, to be typed from the \MA\ interpreter, are related to
console actions.\\
\renewcommand{\arraystretch}{1.2}%
\begin{center}\begin{tabular}{l p{8.8cm}}
\hline
Command & Description\\
\hline
\color{ao} \verb?quit?           & \multirow{2}{*}{Exits \MA.}\\
\color{ao} \verb?EOF?            & \\
\color{ao} \verb?help?           & Displays the list of available commands.\\
\color{ao} \verb?help [command]? & Displays details about a specific command.\\
\color{ao} \verb?history?        & Displays the history of all commands that have been
                        typed by the user.\\
\color{ao} \verb?reset?          & Clears the \MA\ memory as when the program is started.\\
\color{ao} \verb?restart?        & Restarts \MA.\\
\color{ao} \verb?shell [command]?& \multirow{2}{*}{Executes a {\sc Unix} command from the
   interpreter.}\\
\color{ao} \verb?![command]?     & \\
\color{ao} \verb?#[text]?        & Indicates a comment.\\
\hline
\end{tabular}
\end{center}
\verbdef{\verbtexta}{define [(multi)particle label] = [ID1] [ID2] ...}
\verbdef{\verbtextb}{define_region [label1] [label2] ...}
\verbdef{\verbtextc}|plot [obs]([part1] [part2] ...) nbins min max [ [opts] ] { [regs] }|
\verbdef{\verbtextda}{reject ([part]) [criterion]}
\verbdef{\verbtextdb}{select ([part]) [criterion]}
\verbdef{\verbtexte}{[criterion]}
\verbdef{\verbtextf}{swap main.selection[n1] main.selection[n2]}
In order to design an analysis in the \MA\ framework, the user can rely on the
following commands.
\renewcommand{\arraystretch}{1.2}%
\begin{center}\begin{tabular}{l p{8.75cm}}
\hline
Command & Description\\
\hline
\multicolumn{2}{l}{\color{ao} \verbtexta}\\
   & Creates a new (multi)particle label attached to the provided PDG codes
      \verb+[ID1]+, \verb+[ID2]+, $\ldots$\\
\multicolumn{2}{l}{\color{ao} \verbtextb}\\
   & Creates one or more signal regions named \verb+[label1]+, \verb+[label2]+,
   ...\\
\color{ao} \verb?import [label]?        & Imports a sample or a UFO model.\\
\color{ao} \verb?open [folder]?         & Opens a report from the folder \verb?[folder]?.
   If the folder is unspecified, the last created report is open.\\
\multicolumn{2}{l}{\color{ao} \verbtextc}\\
   & Defines a histogram with the distribution in the observable \verb?[obs]?.
    Its computation may require to combine the momenta of several objects
    \verb?[part1]?, \verb+[part2]+, $\ldots$ Any other parameter is optional.
    The \verb?nbins?,
    \verb?min? and \verb?max? quantities respectively correspond to the number
    of bins, and the upper and lower bound of the $x$-axis of the histogram. The
    options \verb?[opts]? (see below) allow the user to tune the
    display of the histogram, and \verb+[regs]+ indicates to which signal region
    one needs to attach this histogram.\\
\color{ao} \verb?reject [criterion]?
\color{ao} \verb?select [criterion]? & Defines a selection cut that leads
  to the rejection (or selection) of an event if the \verbtexte condition is
  satisfied.\\
\multicolumn{2}{l}{\color{ao} \verbtextda}\\
\multicolumn{2}{l}{\color{ao} \verbtextdb}\\
  & Defines a selection cut that leads to the rejection (or selection) of an
    object candidate if the \verb?[criterion]? condition is satisfied.\\
\color{ao} \verb?remove [object]? & Deletes an object (a (multi)particle label, a region, a
 histogram or a cut).\\
\color{ao} \verb?resubmit? & Adjusts the last generated C++ code relatively to the commands
  issued after the last \verb?submit? command, and executes it.\\
\color{ao} \verb?set [obj] = [val]? & Sets an attribute of a specific object to a given
  value.\\
\color{ao} \verb?submit [folder]? & Generates, compiles and executes the C++ code
  corresponding to the current analysis either in a folder named
  \verb?[folder]?, if specified, or in an arbitrary folder otherwise.\\
\multicolumn{2}{l}{\color{ao} \verbtextf}\\ & Swaps the analysis steps
   number \verb?[n1]? and \verb?[n2]?.\\
\hline
\end{tabular}
\end{center}
Throughout the analysis, information can be printed to the screen by means of
the following commands.
\renewcommand{\arraystretch}{1.2}%
\begin{center}\begin{tabular}{l p{7.70cm}}
\hline
Command & Description\\
\hline
\color{ao} \verb+display [object]+       & Displays the properties of a specific object.\\
\color{ao} \verb+display_datasets+       & Displays the list of all defined datasets.\\
\color{ao} \verb+display_multiparticles+ &
   Displays the list of all defined multiparticle labels.\\
\color{ao} \verb+display_particles+      &
   Displays the list of all defined particle labels.\\
\color{ao} \verb+display_regions+        &
   Displays the list of all defined signal regions.\\
\hline
\end{tabular}
\end{center}
\newpage
%%%%%%%%%%%%%%%%%%%%%%%%%%%%%%%%%%%%%%%%%%%%%%%%%%%%%%%%%%%%%%%%%%%%%%%%%%%%%%%%%%%%%%%%%%%%%%
\begin{shaded}
\section{\Large Properties of the main object}
\end{shaded}
%%%%%%%%%%%%%%%%%%%%%%%%%%%%%%%%%%%%%%%%%%%%%%%%%%%%%%%%%%%%%%%%%%%%%%%%%%%%%%%%%%%%%%%%%%%%%%


The \verb+main+ object of \MA\ allows to setup varied options (see the table
below) to impact the code on run time. They can be modified and
displayed by using the \verb+set+ and \verb+display+ commands introduced above.\\

\renewcommand{\arraystretch}{1.2}%
\begin{center}\begin{tabular}{l p{7.50cm}}
\hline
Command & Description\\
\hline
\color{ao} \verb?main.currentdir?     & Folder containing \MA.\\
\color{ao} \verb?main.fastsim.package?& The package used for the fast-simulation of the
  detector response. The available choices are \verb?fastjet?, \verb?delphes?,
  \verb?delphesMA5tune? and \verb?none?.\\
\color{ao} \verb?main.fom.formula?    & Formula to be used to calculate the figure of merit
  in the cutflow charts. Denoting by $S$ and $B$ the number of signal and
  background events, the available choices are 1 ($S/B$), 2 ($S/\sqrt{B}$), 3
  ($S/\sqrt{B}$), 4 ($S/\sqrt{S+B}$) and 5 ($S/\sqrt{S+B+x_B^2}$). For this last
  formula, the $x_B$ parameter is specified via \verb?main.fom.x?.\\
\color{ao} \verb?main.graphic_render? & Package to use for figure generation. The
  available choices are \verb?root?, \verb?matplotlib? and \verb?none?.\\
\color{ao} \verb?main.isolation.algorithm? & Algorithm to be used for particle isolation.
  The available choices are \verb?cone? (no activity in a cone of radius
  specified by \verb+main.isolation.radius+) and \verb+sumpt+ (the scalar sum of
  the transverse momenta of all particles lying in a given cone around the
  candidate must be smaller than \verb?main.isolation.sumPT? and the ratio of
  the transverse energy of all particles in this cone to the transverse momentum
  of the candidate must be smaller than
  \verb?main.isolation.ET_PT?).\\
\end{tabular}
\begin{tabular}{l p{8.4cm}}
\color{ao} \verb?main.lumi? & Integrated luminosity, in $\textrm{fb}^{-1}$, to use for
  histogram and cutflow normalization.\\
\color{ao} \verb?main.normalize? & Way in which histograms have to be normalized. The
  available choices are \verb?none? (each event counts for 1), \verb?lumi?
  (normalization to the integrated luminosity without taking into account the
  event weights), \verb?lumi_weight? (as \verb?lumi? but with the
  event weights).\\
\color{ao} \verb?main.outputfile? & Name of the output file to write events onto.\\
\color{ao} \verb?main.recast? & Switching the recasting mode \verb?on? and \verb?off?.\\
\color{ao} \verb?main.stacking_method? & Way in which the contributions of the different
  datasets to a histogram are displayed. The available choices are
  \verb?normalize2one? (the integral of each contribution equals 1),
  \verb?stack? (each contribution is stacked) and \verb?superimpose? (each
  contribution is superimposed).\\
\hline
\end{tabular}
\end{center}
\newpage 

%%%%%%%%%%%%%%%%%%%%%%%%%%%%%%%%%%%%%%%%%%%%%%%%%%%%%%%%%%%%%%%%%%%%%%%%%%%%%%%%%%%%%%%%%%%%%%
\begin{shaded}
\section{\Large Observables to be used for histograms and cuts}
\label{sec:obs}
\end{shaded}
%%%%%%%%%%%%%%%%%%%%%%%%%%%%%%%%%%%%%%%%%%%%%%%%%%%%%%%%%%%%%%%%%%%%%%%%%%%%%%%%%%%%%%%%%%%%%%


As shown in the tables above, the definition of a histogram or of a cut
condition necessitates to provide an observable that could depend on the momentum
of one or more particles or objects. We list in the tables below all observables
supported by \MA, and begin with those that do not depend on the momenta of any
object. They are therefore called without any argument.\\
\renewcommand{\arraystretch}{1.2}%
\begin{center}\begin{tabular}{l p{10.2cm}}
\hline
Symbol& Description\\
\hline
\color{ao} \verb?ALPHA_QCD? & Value of the QCD coupling constant.\\
\color{ao} \verb?ALPHA_QED? & Value of the electromagnetic coupling constant.\\
\color{ao} \verb?ALPHA_T?   & The $\alpha_T$ variable\cite{Randall:2008rw}.\\
\color{ao} \verb?MEFF?      & Effective mass being defined as the sum of the transverse
                   momentum of all final-state objects and the missing
                   transverse energy.\\
\color{ao} \verb?MET?       & Missing transverse energy.\\
\color{ao} \verb?MHT?       & Missing transverse energy defined from the jet activity
                   only.\\
\color{ao} \verb?NPID?      & Particle content (PDG code distribution).\\
\color{ao} \verb?NAPID?     & Particle content (PDG code distribution in
                   absolute value).\\
\color{ao} \verb?SQRTS?     & Partonic center-of-mass energy.\\
\color{ao} \verb?SCALE?     & Energy scale of the event.\\
\color{ao} \verb?TET?       & Scalar sum of the transverse energy of all final-state
                   objects.\\
\color{ao} \verb?THT?       & Scalar sum of the transverse energy of all final-state
                   jets.\\
\color{ao} \verb?WEIGHTS?   & Event weights.\\
\hline
\end{tabular}
\end{center}
The set of observables provided in the following table can be used to study the
properties of a given object or particle, and thus requires to provide one
four-momentum or one combination of four-momenta as an argument of the
observable function.
\renewcommand{\arraystretch}{1.2}%
\begin{center}\begin{tabular}{l p{10.2cm}}
\hline
Symbol& Description\\
\hline
\color{ao} \verb?ABSETA? & Absolute value of the pseudorapidity.\\
\color{ao} \verb?BETA?   & Velocity $\beta=v/c$ (relatively to the speed of light).\\
\color{ao} \verb?E?      & Energy.\\
\color{ao} \verb?EE_HE?  & Ratio of the electromagnetic energy to the hadronic energy (for
   a jet).\\
\color{ao} \verb?ET?     & Transverse energy.\\
\color{ao} \verb?ETA?    & Pseudorapidity.\\
\color{ao} \verb?GAMMA?  & Lorentz-factor.\\
\color{ao} \verb?HE_EE?  & Ratio of the hadronic energy to the electromagnetic energy (for
   a jet).\\
\color{ao} \verb?M?      & Invariant mass.\\
\color{ao} \verb?MT?     & Transverse mass.\\
\color{ao} \verb?MT_MET? & Transverse mass of the system comprised of the object and the
   missing momentum.\\
\color{ao} \verb?NTRACKS?& Number of tracks (inside a jet).\\
\color{ao} \verb?P?      & Magnitude of the three-momentum.\\
\color{ao} \verb?PHI?    & Azimuthal angle.\\
\color{ao} \verb?PT?     & Transverse momentum.\\
\color{ao} \verb?PX?     & $x$-component of the momentum.\\
\color{ao} \verb?PY?     & $y$-component of the momentum.\\
\color{ao} \verb?PZ?     & $z$-component of the momentum.\\
\color{ao} \verb?R?      & Position in the $(\eta, \phi)$ plane.\\
\color{ao} \verb?Y?      & Rapidity.\\
\hline
\end{tabular}
\end{center}
\noindent Three additional observables involving two objects are also available, the
methods given in the table below taking thus two arguments separated by a comma.
\renewcommand{\arraystretch}{1.2}%
\begin{center}\begin{tabular}{l p{10.0cm}}
\hline
Symbol& Description\\
\hline
\color{ao} \verb?DELTAR?     & Angular distance, in the transverse plane, between the
  objects.\\
\color{ao} \verb?DPHI_0_PI?  & Angular distance in azimuth between the objects. The
  bounds for the angle are $[0,\pi]$.\\
\color{ao} \verb?DPHI_0_2PI? & Angular distance in azimuth between the objects. The
  bounds for the angle are $[0,2\pi]$.\\
\hline
\end{tabular}
\end{center}
\noindent For all arguments of any of the above observable, any sequence of momenta
separated with spaces will lead to a sum of these momenta before computing the
observable. For instance,
{\color{ao} \begin{verbatim}
  plot M(e+ e-)
\end{verbatim}}
\noindent allows for the computation of the invariant of an electron-positron system.
\newpage
%%%%%%%%%%%%%%%%%%%%%%%%%%%%%%%%%%%%%%%%%%%%%%%%%%%%%%%%%%%%%%%%%%%%%%%%%%%%%%%%%%%%%%%%%%%%%%
\begin{shaded}
\section{\Large Options for histograms}
\end{shaded}
%%%%%%%%%%%%%%%%%%%%%%%%%%%%%%%%%%%%%%%%%%%%%%%%%%%%%%%%%%%%%%%%%%%%%%%%%%%%%%%%%%%%%%%%%%%%%%

\noindent The command \verb+plot+ accept varied options \verb+[opts]+,
{\color{ao} \begin{verbatim}
plot [obs]([part1] [part2] ...) nbins min max [ [opts] ] { [regs] }
\end{verbatim}}
to be provided between squared brackets. The list of all available choices is
given in the table below.
\renewcommand{\arraystretch}{1.2}%
\begin{center}\begin{tabular}{l p{9.4cm}}
\hline
Symbol& Description\\
\hline
\color{ao} \verb?Eordering?     & Sorts the objects in increasing energy.\\
\color{ao} \verb?ETordering?    & Sorts the objects in increasing transverse energy.\\
\color{ao} \verb?ETAordering?   & Sorts the objects in increasing pseudorapidity.\\
\color{ao} \verb?Pordering?     & Sorts the objects in increasing three-momentum magnitude.
  \\
\color{ao} \verb?PTordering?    & Sorts the objects in increasing transverse momentum.\\
\color{ao} \verb?PXordering?    & Sorts the objects in increasing momentum $x$-component.\\
\color{ao} \verb?PYordering?    & Sorts the objects in increasing momentum $y$-component.\\
\color{ao} \verb?PZordering?    & Sorts the objects in increasing momentum $z$-component.\\
\color{ao} \verb?allstate?      & Considers all (initial-state, final-state and
  intermediate-state) objects in the events.\\
\color{ao} \verb?finalstate?    & Considers only final-state objects (default).\\
\color{ao} \verb?initialstate?  & Considers only initial-state objects.\\
\color{ao} \verb?interstate?    & Considers only objects that are neither initial-state nor
  final-state objects.\\
\color{ao} \verb?logX?          & Logarithmic scale for the $x$-axis.\\
\color{ao} \verb?logY?          & Logarithmic scale for the $y$-axis.\\
\color{ao} \verb?normalize2one? & Normalizes the histogram to 1.\\
\color{ao} \verb?stack?         & Stacks the contributions of different datasets in the
  histogram.\\
\color{ao} \verb?superimpose?   & Superimposes the contributions of different datasets in
  the histogram.\\
\hline
\end{tabular}
\end{center}

\vspace{2cm}
%%%%%%%%%%%%%%%%%%%%%%%%%%%%%%%%%%%%%%%%%%%%%%%%%%%%%%%%%%%%%%%%%%%%%%%%%%%%%%%%%%%%%%%%%%%%%%
\begin{shaded}
\section{\Large Options for datasets}
\label{sec:datasets}
\end{shaded}
%%%%%%%%%%%%%%%%%%%%%%%%%%%%%%%%%%%%%%%%%%%%%%%%%%%%%%%%%%%%%%%%%%%%%%%%%%%%%%%%%%%%%%%%%%%%%%

\noindent Once one or more samples have been imported as a dataset, properties that will
impact the display of the subsequent contribution in histograms can be modified
by using the \verb+set+ command. For instance, a dataset named \verb?defaultset?
can be imported as
{\color{ao} \begin{verbatim}
  import <path-to-sample> as defaultset
\end{verbatim}}
\noindent and its properties can be modified by typing in
{\color{ao} \begin{verbatim}
  set defaulset.<property> = <value>
\end{verbatim}}
\noindent The list of available properties is given in the table below, together with the
allowed values.
\renewcommand{\arraystretch}{1.2}%
\begin{center}\begin{tabular}{l p{9.1cm}}
\hline
Symbol& Description\\
\hline
\color{ao} \verb?backcolor? & Background color in a histogram. The available colors are
   \verb?auto?, \verb?black?, \verb?blue?, \verb?cyan?, \verb?green?,
   \verb?grey?, \verb?none? (transparent), \verb?orange?, \verb?purple?,
   \verb?red?, \verb?white? and \verb?yellow?. The color can be made lighter or
    darker by adding an explicit $\pm 1$, $\pm 2$ or $\pm 3$.\\
\color{ao} \verb?backstyle? & Background texture in a histogram. The available values are
   \verb?dline? (diagonal-lines), \verb?dotted? (dots), \verb?hline? (horizontal
   lines), \verb?solid? (uniform color) and \verb?vline? (vertical lines).\\
\color{ao} \verb?linecolor? & Color of the histogram lines. The available colors are the
   same as for the \verb?backcolor? attribute.\\
\color{ao} \verb?linestyle? & Style of the histogram lines. The available values are
   \verb?dash-dotted?, \verb?dashed?, \verb?dotted? and \verb?solid?.\\
\color{ao} \verb?linewidth? & Width of the histogram lines, given as an integer smaller
   than 10.\\
\color{ao} \verb?title?     & Name of the dataset (for histogram legends).\\
\color{ao} \verb?type?      & \verb?background? or \verb?signal? nature of a given sample
  (for figure-of-merit calculations).\\
\color{ao} \verb?weight?    & Reweights each histogram entry with a constant factor. The
  value has to be a floating-point number.\\
\color{ao} \verb?weighted_events? & Allows \MA\ to ignore the weights of the events
  (property to be set to \verb?true? or \verb?false?).\\
\color{ao} \verb?xsection?  & This overwrites the event sample cross section. The value has
  to be given in pb.\\
\color{ao} \verb?scale_variation? & This attaches to the dataset cross section
  a theory error associated with the variation of the unphysical scales. The
  value has to lie in the [0,1] interval and the error band is symmetric around
  the central value. \\
\color{ao} \verb?pdf_variation?   &  This attaches to the dataset cross section
  a theory error associated with the variation of the parton densities. The
  value has to lie in the [0,1] interval and the error band is symmetric around
  the central value. \\
\color{ao} \verb?scale_up_variation? & This attaches to the dataset cross section
  a theory error associated with the variation of the unphysical scales. The
  value has to lie in the [0,1] interval and this parameter concerns the upper
  enveloppe of the error band.\\
\color{ao} \verb?scale_down_variation? & This attaches to the dataset cross section
  a theory error associated with the variation of the unphysical scales. The
  value has to lie in the [0,1] interval and this parameter concerns the lower
  enveloppe of the error band.\\
\color{ao} \verb?pdf_up_variation? &
  This attaches to the dataset cross section
  a theory error associated with the variation of the parton densities. The
  value has to lie in the [0,1] interval and this parameter concerns the upper
  enveloppe of the error band.\\
\color{ao} \verb?pdf_down_variation? &
  This attaches to the dataset cross section
  a theory error associated with the variation of the parton densities. The
  value has to lie in the [0,1] interval and this parameter concerns the lower
  enveloppe of the error band.\\
\hline
\end{tabular}
\end{center}

\vspace{2cm}

%%%%%%%%%%%%%%%%%%%%%%%%%%%%%%%%%%%%%%%%%%%%%%%%%%%%%%%%%%%%%%%%%%%%%%%%%%%%%%%%%%%%%%%%%%%%%%
\begin{shaded}
\section{\Large Using \FJnorm\ through \MAnorm}\label{sec:fastjet}
\end{shaded}
%%%%%%%%%%%%%%%%%%%%%%%%%%%%%%%%%%%%%%%%%%%%%%%%%%%%%%%%%%%%%%%%%%%%%%%%%%%%%%%%%%%%%%%%%%%%%%

\noindent In order to activate the usage of \FJ\ through \MA, the program has to be
started in the reconstructed mode and the first command to be typed in the
interpreter has to be
{\color{ao} \begin{verbatim}
  set main.fastsim.package = fastjet
\end{verbatim}}
\noindent This allows for various option of the \verb+main.fastsim+ object, tuning the
properties of the jet algorithm that has to be employed. Those options are set
by typing in
{\color{ao} \begin{verbatim}
  set main.fastsim.<property> = <value>
\end{verbatim}}
\noindent the list of all available properties being presented, together with the allowed
values, in the following table.
\renewcommand{\arraystretch}{1.2}%
\begin{center}\begin{tabular}{l p{9.4cm}}
\hline
Symbol& Description\\
\hline
\color{ao} \verb?algorithm? & Sets up the jet algorithm to use. The allowed
  values are \verb?antikt?\cite{Cacciari:2008gp}, \verb?Cambridge?\cite{%
   Dokshitzer:1997in,Wobisch:1998wt}, \verb?cdfjetclu?\cite{Abe:1991ui},
   \verb?cdfmidpoint?\cite{Blazey:2000qt}, \verb?genkt?\cite{Cacciari:2011ma},
   \verb?gridjet?\cite{Cacciari:2011ma}, \verb?kt?\cite{Catani:1993hr,%
   Ellis:1993tq}, \verb?none? and \verb?siscone?\cite{Salam:2007xv}.\\
\color{ao} \verb?areafraction? & Controls the size of the cones in the CDF midpoint
  algorithm.\\
\color{ao} \verb?exclusive_id? & Exclusive mode for jet reconstruction. If set
  to \verb+false+, electrons muons and photons issued from hadron decays are
  included into the electron, muon and photon collections respectively.\\
\color{ao} \verb?input_ptmin? & Soft protojet threshold in the siscone
  algorithm.\\
\color{ao} \verb?iratch? & Switching on ratcheting for the CDF jet clustering algorithm.\\
\color{ao} \verb?npassmax? & Number of iterations in the siscone algorithm.\\
\color{ao} \verb?overlap? & Fraction of overlapping momentum used to combine
  protojets in the siscone and CDF reconstruction algorithms.\\
\color{ao} \verb?p? & $p$ parameter of the generalized $k_T$ algorithm.\\
\color{ao} \verb?ptmin? & Threshold for the transverse momentum of the
  reconstructed jets.\\
\color{ao} \verb?radius? & Radius parameter relevant for most jet
  clustering algorithms.\\
\color{ao} \verb?seed? & Threshold parameter used in the constituent merging
  procedure of the CDF reconstruction algorithms.\\
\color{ao} \verb?spacing? & Grid spacing in the grid jet algorithm.\\
\color{ao} \verb?ymax? & Maximum rapidity value in the grid jet algorithm.\\
\hline
\end{tabular}
\end{center}
\vspace{0.1cm}
\noindent In addition, basic detector simulation effects can be mimicked through setting
up the parameters included in the table below.
\renewcommand{\arraystretch}{1.2}%
\begin{center}\begin{tabular}{l p{8.4cm}}
\hline
Symbol& Description\\
\hline
\color{ao} \verb?bjet_id.efficiency? & $b$-tagging efficiency, as a float.\\
\color{ao} \verb?bjet_id.exclusive? & Allows several $b$-jets to be issued from a single
  $B$-hadron.\\
\color{ao} \verb?bjet_id.matching_dr? & Angular distance parameter matching a $b$-jet
  with a $B$-hadron.\\
\color{ao} \verb?bjet_id.misid_cjet? & Mistagging rate of a $c$-jet as a $b$-jet, as
  a float.\\
\color{ao} \verb?bjet_id.misid_ljet? & Mistagging rate of a light jet as a $b$-jet,
  as a float.\\
\color{ao} \verb?tau_id.efficiency? & Tau-tagging efficiency, as a float.\\
\color{ao} \verb?tau_id.misid_ljet? &  Mistagging rate of a light jet as a hadronic tau,
   as a float.\\
\hline
\end{tabular}
\end{center}

\vspace{2cm}

%%%%%%%%%%%%%%%%%%%%%%%%%%%%%%%%%%%%%%%%%%%%%%%%%%%%%%%%%%%%%%%%%%%%%%%%%%%%%%%%%%%%%%%%%%%%%%
\begin{shaded}
\section{\Large Using \DELnorm\ through \MAnorm}
\end{shaded}
%%%%%%%%%%%%%%%%%%%%%%%%%%%%%%%%%%%%%%%%%%%%%%%%%%%%%%%%%%%%%%%%%%%%%%%%%%%%%%%%%%%%%%%%%%%%%%


\noindent In order to activate the usage of \DEL\ through \MA, the program has to be
started in the reconstructed mode and the first command to be typed in the
interpreter has to be
{\color{ao}\begin{verbatim}
  set main.fastsim.package = delphes
\end{verbatim}}
\noindent The properties of the simulation of the detector can then be adjusted through
{\color{ao} \begin{verbatim}  set main.fastsim.<property> = <value>
\end{verbatim}}
the allowed choices being given in the table below.
\renewcommand{\arraystretch}{1.2}%
\begin{center}\begin{tabular}{l p{8.4cm}}
\hline
Symbol& Description\\
\hline
\color{ao} \verb?detector? & Determines which detector card to use. The card can be further
  modified on run time).\\
\color{ao} \verb?output?   & Saves the output \ROOT\ file (\verb+true+ or \verb+false+).\\
\color{ao} \verb?rootfile? & Name of the output file.\\
\color{ao} \verb?pileup? & Specifies the path to the input pile-up event file.\\
\color{ao} \verb?skim_genparticles? & If set to \verb?true?, the generator-level
  particles are not stored in the output file.\\
\color{ao} \verb?skim_tracks? & If set to \verb?true?, the track collection is not stored
  in the output file.\\
\color{ao} \verb?skim_towers? & If set to \verb?true?, the collection of calorimetric
  towers is not stored in the output file.\\
\color{ao} \verb?skim_eflow? & If set to \verb?true?, the collection of particle-flow
  towers is not stored in the output file.\\
\hline
\end{tabular}
\end{center}

\vspace{2cm}

%%%%%%%%%%%%%%%%%%%%%%%%%%%%%%%%%%%%%%%%%%%%%%%%%%%%%%%%%%%%%%%%%%%%%%%%%%%%%%%%%%%%%%%%%%%%%%
\begin{shaded}
\section{\Large Multipartonic matrix element merging}
\end{shaded}
%%%%%%%%%%%%%%%%%%%%%%%%%%%%%%%%%%%%%%%%%%%%%%%%%%%%%%%%%%%%%%%%%%%%%%%%%%%%%%%%%%%%%%%%%%%%%%

\noindent \MA\ can be used to double check the merging procedure of event samples related
to a given hard process but with matrix elements featuring a different
final-state jet multiplicity. \MA\ has to be started in the hadronic mode, and
the check is then performed by typing in
{\color{ao}\begin{verbatim}
 set main.merging.check = true
\end{verbatim}}
\noindent Two extra options are available,
{\color{ao}\begin{verbatim}
  set main.merging.ma5_mode = <true or false>
  set main.merging.njets = <integer>
\end{verbatim}}
\noindent the first one indicating to extract the number of extra hard jets from the
process identifier, and the second one setting up the maximum number of extra
jets to consider.


%%%%%%%%%%%%%%%%%%%%%%%%%%%%%%%%%%%%%%%%%%%%%%%%%%%%%%%%%%%%%%%%%%%%%%%%%%%%%%%%%%%%%%%%%%%%%%
\begin{shaded}
	\section{\Large Super-fast detector simulation with \MAnorm}
\end{shaded}
%%%%%%%%%%%%%%%%%%%%%%%%%%%%%%%%%%%%%%%%%%%%%%%%%%%%%%%%%%%%%%%%%%%%%%%%%%%%%%%%%%%%%%%%%%%%%%
\MA\ offers the user options to parameterize a super-fast simulation of a
detector response when event reconstruction is achieved through \FJ. The user
should first initialize \FJ, through \MA, as shown in~\autoref{sec:fastjet}.
However, as more advanced $b$-tagging and tau-tagging capabilities are
available, default values for all tagging options of the \FJ\ interface must be
enforced to avoid redundancies.

The super-fast detector simulation module (denoted below as the super-fastsim
package of \MA) offers two ways to treat jets, which can be set as
{\color{ao} \begin{verbatim}
  set main.fastsim.jetrecomode = <value>
\end{verbatim}}
\noindent Two choices exist for the value \verb+<value>+ of the
{\tt jetrecomode} attribute,
\renewcommand{\arraystretch}{1.2}%
\begin{center}
  \begin{tabular}{l p{8.4cm}}
    \hline
    Symbol& Description\\
    \hline
    \color{ao} \verb?jets? &
      The jet reconstruction probability distribution and smearing functions are
      applied after the clustering (default).\\
    \color{ao} \verb?constituents?   &
      The jet reconstruction probability distribution and smearing functions are
      applied to each of the jet constituents before their clustering with \FJ.\\
    \hline
  \end{tabular}
\end{center}

\vspace{2cm}

\noindent The super-fastsim package of \MA\ is shipped with three
submodules, for which dedicated examples will be provided below. Their
respectively address object reconstruction, the smearing of the momenta of the
various reconstructed objects and the tagging of a given object as such (tagging
efficiencies) or as a different object (mistagging rates). These modules are
\renewcommand{\arraystretch}{1.2}%
\begin{center}\begin{tabular}{l p{8.4cm}}
  \hline
  Symbol& Description\\
  \hline
  \color{ao} \verb?reco_efficiency? & Generates a probability
    distribution relying on a user-defined piecewise function. This distribution
    describes the efficiency to reconstruct a given object.\\
  \color{ao} \verb?smearer?  &  Generates a Gaussian smearing function relying
    on a user-defined piecewise function that reflects the standard deviation of
    the Gaussian. The four-momentum of the object that the function is attached
    to is then smeared.\\
  \color{ao} \verb?tagger?   & Generates a probability distribution relying on a
    user-defined piecewise function describing the probabily to (mis)tag an
    object. \\
  \hline
 \end{tabular}
\end{center}
\noindent The syntax to be employed to make use of the \verb+reco_efficiency+
submodule reads
{\color{ao} \begin{verbatim}
  define reco_efficiency <obj> <function> [<domain>]
\end{verbatim}}
\noindent In this expression, \verb+<obj>+ consists in the object for which the
reconstruction efficiency is given. The {\tt reco\_efficiency} submodule can be
used for for the reconstruction of jets (\verb+j+), hadronic taus (\verb+ta+),
muons (\verb+mu+),
electrons (\verb+e+) and photons (\verb+a+). The piecewise function on which the
objet reconstruction probability distribution is built from is given through the
{\tt <function>} and {\tt <domain>} arguments, the former being mandatory and
the latter optional. One call to {\tt define reco\_efficiency} has to be
made for each subdomain of the piecewise function. The function {\tt <function>}
can depend on any observable supported by \MA\ (see~\autoref{sec:obs}). All
methods implemented in the {\tt cmath} {\sc Python} package are supported, so
that the code understands for instance all trigonometric, logarithmic and
cyclometric functions. Numbers can be inputted either in their traditional form
or following the scientific format. The domain {\tt <domain>} should contains
one or more inequalities (connected by the logical operators {\tt and} and {\tt
or}) and the analytical form of the inequalities can rely on any function
(supported by the {\tt cmath} {\sc Python} package) of the observables available
in \MA. In the case where the domain is not provided, the efficiency is
understood to be applicable in all cases.

\noindent The syntax to be employed to make use of the \verb+smearer+ submodule
reads
{\color{ao} \begin{verbatim}
  define smearer <obj> with <comp> <function> [<domain>]
\end{verbatim}}
\noindent In this expression, \verb+<obj>+ consists in the object whose momentum
has to be smeared. The {\tt smearer} submodule can be used for for jets
(\verb+j+), hadronic taus (\verb+ta+), muons (\verb+mu+), electrons (\verb+e+)
and photons (\verb+a+). The smearing proceeds by modifying the component of the
object four-momentum {\tt <comp>} according to a Gaussian distribution specified
by the user. In practice, the mean of the distribution is taken to be the
initial value of {\tt <comp>} and its standard deviation stems form the
piecewise function provided by the user {\tt <function>}. As above,
the {\tt smearer} module has to be called for each of the subdomain of the
function. The syntax for providing the function and the domain information is
identical to what has been described for the {\tt reco\_efficiency} module. The
momentum component {\tt <comp>} can be the energy (\verb|E|), the transverse
momentum (\verb|PT|) or any of the three-momentum components (\verb+PX+,
\verb+PY+ and \verb+PZ+) of the smeared object.

\noindent The \verb+tagger+ submodule can be used following the syntax
{\color{ao} \begin{verbatim}
  define tagger <true_obj > as <reco_obj> <function> [<domain>]
\end{verbatim}}
\noindent This indicates to the super-fastsim how to tag an object
{\tt <true\_obj>} as an object {\tt <reco\_obj>} with a given probability
specified by a piecewise function. This function and its domain are provided
according to the rules described above, and the allowed tagging and mistagging
pairs are given in the table below.
\renewcommand{\arraystretch}{1.2}%
\begin{center}\begin{tabular}{l p{8.4cm}}
  \hline
  Symbol& Description\\
  \hline
  \color{ao} \verb?b as b? & Efficiency to tag a true $b$-jet as a $b$-jet.\\
  \color{ao} \verb?c as b? & Mistagging rate of a true $c$-jet as a $b$-jet.\\
  \color{ao} \verb?j as b? & Mistagging rate of a true light jet as a $b$-jet.\\
  \color{ao} \verb?c as c? & Efficiency to tag a true $c$-jet as a $c$-jet.\\
  \color{ao} \verb?j as c? & Mistagging rate of a true light jet as a $c$-jet.\\
  \color{ao} \verb?b as c? & Mistagging rate of a true $b$-jet as a $c$-jet.\\
  \color{ao} \verb?ta as ta? & Efficiency to tag a true hadronic tau as a
    hadronic tau.\\
  \color{ao} \verb?j as ta? & Mistagging rate of a true light jet as a hadronic
    tau.\\
  \color{ao} \verb?e as j?  & Mistagging rate of a true electron as a jet.\\
  \color{ao} \verb?e as mu? & Mistagging rate of a true electron as a muon.\\
  \color{ao} \verb?e as a?  & Mistagging rate of a true electron as a photon.\\
  \color{ao} \verb?mu as j? & Mistagging rate of a true muon as a jet.\\
  \color{ao} \verb?mu as e? & Mistagging rate of a true muon as an electron .\\
  \color{ao} \verb?mu as a? & Mistagging rate of a true muon as a photon.\\
  \color{ao} \verb?a as j?  & Mistagging rate of a true photon as a jet.\\
  \color{ao} \verb?a as e?  & Mistagging rate of a true photon as an electron .\\
  \color{ao} \verb?a as mu? & Mistagging rate of a true photon as a muon.\\
  \hline
 \end{tabular}
\end{center}
\vspace{2cm}
%%%%%%%%%%%%%%%%%%%%%%%%%%%%%%%%%%%%%%%%%%%%%%%%%%%%%%%%%%%%%%%%%%%%%%%%%%%%%%%%%%%%%%%%%%%%%%
\begin{shaded}
\section{\Large Recasting LHC analyses with \MA}
\end{shaded}
%%%%%%%%%%%%%%%%%%%%%%%%%%%%%%%%%%%%%%%%%%%%%%%%%%%%%%%%%%%%%%%%%%%%%%%%%%%%%%%%%%%%%%%%%%%%%%
A collection of LHC analyses have been reimplemented in the \MA\ framework and
are available from the public page
\href{http://madanalysis.irmp.ucl.ac.be/wiki/PublicAnalysisDatabase}
{http://madanalysis.irmp.ucl.ac.be/wiki/PublicAnalysisDatabase}. The so-called
recasting module of \MA\ allows one to download a commented C++ code associated
with each of these analyses, together with information allowing to extract
bounds on a given new physics signal. This download is achieved by typing in
{\color{ao} \begin{verbatim}
  install delphes
  install delphesMA5tune
  install PAD
  install PADForMA5tune
  install PADForSFS
\end{verbatim}}
\noindent The first two commands trigger the installation of the current release
of \DEL\ as well as of its old tuned version, whilst the last three commands
install the three versions of the \MA\ Public Analysis Database of recasted LHC
analyses. The first version ({\tt PAD}) relies on the current release of \DEL\
for the simulation of the detector reponse, whilst the second one ({\tt
PADForMA5tune}) relies on the
tuned version of \DEL. The third framework ({\tt PADForSFS}) uses the Simplified
Fast Simulation (SFS) module of \MA\ to deal with the simulation of the
detector response. More information can be found in~\autoref{sec:install}.
In order to switch on the recasting module of \MA, the user needs to type in
{\color{ao} \begin{verbatim}
  set main.recast = on
\end{verbatim}}
\noindent the defaul value being \verb|off|. The user must then import a
hadron-level dataset stored under the HepMC or StdHep format, as detailed in
\autoref{sec:datasets}. The options allowing to setup the dataset cross section,
together with the associated theory uncertainties, are crucial for
the good functioning of the recasting module and the extraction of a correct
limit on the signal.

\noindent The module moreover allows the user to add systematic uncertainties on
the signal and extrapolate the results to differnt luminosities by typing in
{\color{ao} \begin{verbatim}
  set main.recast.add.<property> = <value>
\end{verbatim}}
\noindent where the implemented properties and possible values are given below.
\renewcommand{\arraystretch}{1.2}%
\begin{center}\begin{tabular}{l p{8.4cm}}
\hline
  Symbol& Description\\
  \hline
  \color{ao} \verb?systematics? & Adds systematic uncertainty to the signal, the
    value of the attribute being in the [0,1] interval. A single value indicates
    that the uncertainties are symmetric and two values separated by a space or
    a comma indicate asymmetric errors. The user can add several systematic
    uncertainties that will be combined in quadrature. \\
  \color{ao} \verb?extrapolated_luminosity?  & Indicates that the results will
    have to be extrapolated to other lumiosities, to be provided in fb$^{-1}$
    and as floating-point numbers. Multiple luminosities can be added, separated
    by a space or a comma. A separate output is generated for each luminosity
    value.\\
  \hline
\end{tabular}
\end{center}

\noindent The module offers the user a few additional properties that can be set
as
{\color{ao} \begin{verbatim}
  set main.recast.<property> = <value>
\end{verbatim}}
The list of available options is given below.
\renewcommand{\arraystretch}{1.2}%
\begin{center}\begin{tabular}{l p{8.4cm}}
\hline
  Symbol& Description\\
  \hline
  \color{ao} \verb?CLs_numofexps? & Number of toy experiments to be used in the
    CLs calculations (the default value is 100000). \\
  \color{ao} \verb?card_path?     & Path of the recasting card containing the
    list of analyses to reinterpret (if not provided, a default card is
    generated ny \MA).\\
  \color{ao} \verb?store_root?    & Boolean indicatign whether the root file
    outputted by \DEL\ should be stored (the default value is \verb|false|).\\
  \color{ao} \verb?THerror_combination? & Indicates how the theoretical
    uncertainties should be combined, \ie, either quadratically
    (\verb|quadratic|) or linearly (\verb|linear|, default).\\
  \color{ao} \verb?error_extrapolation? & Indicates how the uncertainties should
    be scaled when the results are extrapolated to other luminosities. The
    possible options are \verb|linear| (\ie, as for systematic uncertainties;
    default) or \verb|sqrt| (\ie, as for statistical uncertainties).\\
\hline
\end{tabular}
\end{center}




\newpage
\bibliographystyle{plain} 
\bibliography{main_normal}

\end{document}
